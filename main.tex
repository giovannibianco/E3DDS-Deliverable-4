\documentclass[12pt,a4paper]{article}
\PassOptionsToPackage{table}{xcolor}
% \documentclass{report}
\usepackage[utf8]{inputenc}
\usepackage{xspace}
\usepackage{url}
\usepackage{hyperref}
\usepackage{fancyhdr}
\usepackage{cite}
\usepackage{pgfgantt}
\usepackage{todonotes}
\usepackage[icelandic,UKenglish]{babel}
\usepackage[UKenglish]{datetime}
\usepackage[T1]{fontenc}
\usepackage{graphicx}
\usepackage[table]{xcolor}
\usepackage{enumitem}% http://ctan.org/pkg/enumitem
\usepackage[gen]{eurosym}
\usepackage{multicol}
\graphicspath{ {./images/} }
\usepackage[titletoc,title]{appendix}
\usepackage{pdfpages}
% \usepackage{array}
% \usepackage{pbox}
% \usepackage{hanging}
\usepackage{fancyhdr}
% \usepackage{caption}
% \usepackage{enumitem}
\usepackage{fancyvrb}
\usepackage{color}
\usepackage[utf8]{inputenc}
\usepackage{graphicx}
\usepackage{listings}
\usepackage{url}

\usepackage{xspace}

\usepackage{rotating}
\usepackage{wrapfig}

\usepackage{cite}
\usepackage{textcomp}

\usepackage[gen]{eurosym}
\usepackage{hyperref}

\usepackage{todonotes}

\usepackage{gensymb}

% \usepackage{draftwatermark}
% \SetWatermarkText{draft v2}
% \SetWatermarkScale{6}

\fancypagestyle{plain}{ %
  \fancyhf{} % remove everything
  \renewcommand{\headrulewidth}{0pt} % remove lines as well
  \renewcommand{\footrulewidth}{0pt}
}

\hyphenation{EISCAT}

\newenvironment{MYitemize}{%
    %% \renewcommand{\labelitemi}{$\rightarrow$}%
    %% \renewcommand{\labelitemii}{$\circ$}%
    %% \renewcommand{\labelitemii}{$\rightarrow$}%
    %% \renewcommand{\labelitemiii}{$\rightarrow$}%
    %% \renewcommand{\labelitemiii}{\colblack $\cdot$}%
    \begin{itemize}}{\end{itemize}}

\newcommand{\bitm}{\begin{MYitemize}}
\newcommand{\eitm}{\end{MYitemize}}

\newcommand{\N}{NeIC\xspace}
\newcommand{\NP}{NeIC~E3DDS\xspace} % NeIC project

\newcommand{\ED}{EISCAT\_3D\xspace}
\newcommand{\EC}{EISCAT\xspace}
\newcommand{\ESA}{EISCAT Scientific Association\xspace}
\newcommand{\SA}{sub-array\xspace}
\newcommand{\SAs}{sub-arrays\xspace}
\newcommand{\RS}{radar site\xspace}
\newcommand{\RSs}{radar sites\xspace}
\newcommand{\OS}{on-site\xspace}
\newcommand{\OC}{operations centre\xspace}
\newcommand{\CC}{control centre\xspace}
\newcommand{\DC}{data centre\xspace}
\newcommand{\DCa}{dCache\xspace}
\newcommand{\DCs}{data centres\xspace}
\newcommand{\UAF}{User Analysis Facility\xspace}

\newcommand{\RB}{ring buffer\xspace}
\newcommand{\fsru}{first stage receive unit\xspace}
\newcommand{\FW}{file writer\xspace}
\newcommand{\SBF}{second beam former\xspace}
\newcommand{\NNB}{$2 \times 100$\xspace} %number of narrow beams

\newcommand{\NBW}{5~MHz\xspace} % narrow bandwidth
\newcommand{\WBW}{30~MHz\xspace} % wide bandwidth

\newcommand{\ramf}{Ramfjordmoen\xspace}
\newcommand{\controllatency}{less than $5$\xspace}
\newcommand{\CP}{C/C++\xspace}
\newcommand{\DI}{DIRAC\xspace}
\newcommand{\DRF}{DigitalRF\xspace}
\newcommand{\HDF}{HDF5\xspace}
\newcommand{\IB}{Infiniband\xspace}
\newcommand{\IBtwo}{200\,Gb/s~\IB}
\newcommand{\Ru}{Rucio\xspace}
\newcommand{\einfra}{e-infrastructure\xspace}
\newcommand{\neinfra}{national e-infrastructure\xspace}
\newcommand{\WR}{White Rabbit\xspace}
\newcommand{\UAP}{User access portal\xspace}

\newcommand{\bfmust}{\xspace{\bf must}\xspace}
\newcommand{\bfshould}{\xspace{\bf should}\xspace}
\newcommand{\Bfmust}{\xspace{\bf Must}\xspace}
\newcommand{\Bfshould}{\xspace{\bf Should}\xspace}
\newcommand{\bfshall}{\xspace{\bf shall}\xspace}
\newcommand{\Bfshall}{\xspace{\bf Shall}\xspace}

\newcommand{\FirstBF}{\xspace{\bf Shall}\xspace}

\newcommand{\visroot}{isualiz}
\newcommand{\vis}{\visroot{ation}\xspace}
\newcommand{\vise}{\visroot{e}\xspace}
% \newcommand{\Vis}{{Visualization}\xspace}
\newcommand{\threed}{{3-dimensional}\xspace}
\newcommand{\GD}{{GUISDAP}\xspace}

\newcommand{\gps}{{Gbit/s}\xspace}

\newcommand{\np}{national provider\xspace}
\newcommand{\nps}{\np{s}\xspace}

\newcommand\EatDot[1]{}

\newcommand{\pc}{prompt computing\xspace}

%\newcommand{\NBW}{5~MHz} % narrow bandwidth
%\newcommand{\WBW}{30~MHz} % wide bandwidth


% \setlength{\topskip}{0mm}
\setlength{\headheight}{15pt}
% \setlength{\topmargin}{-5.4mm}
% \setlength{\textheight}{230mm}
\setlength{\textwidth}{180mm}
\setlength{\oddsidemargin}{-5.0mm}
% \setlength{\evensidemargin}{10.0mm}
% \setlength{\captionmargin}{7mm}

\title{
{\bf Deliverable Document 3} \\
Deployment of \ED data processing simulation}
\author{E3DDS Team~\footnote{
Anders Tjulin (EISCAT) {\tt anders.tjulin@eiscat.se};
Ari Lukkarinen (CSC) {\tt ari.lukkarinen@csc.fi};
Assar Westman (EISCAT) {\tt Assar.Westman@eiscat.se};
Carl-Fredrik Enell (EISCAT) {\tt carl-fredrik.enell@eiscat.se};
Dan Johan Jonsson (UiT) {\tt dan.jonsson@uit.no};
Janos Nagy (NSC) {\tt fconagy@nsc.liu.se};
Harri Hellgren (EISCAT) {\tt harri.hellgren@eiscat.se};
Ingemar H\"{a}ggstr\"{o}m (EISCAT) {\tt ingemar.haggstrom@eiscat.se};
Mattias Wadenstein (UmU) {\tt maswan@hpc2n.umu.se};
% Roy Dragseth (UiT) {\tt roy.dragseth@uit.no};
John White (NeIC) {\tt john.white@cern.ch}}}

\date{\today}

\begin{document}

\pagestyle{fancy}
\lhead{\bf E3DDS project}
\rhead{\bf 4: Data chain simulation}

\maketitle
\par\noindent
\begin{minipage}{0.5\textwidth}
  \includegraphics[scale=0.18]{NEIC_logo_screen_black.pdf}
  %\vspace{-0.09in}
\end{minipage}
\begin{minipage}{0.5\textwidth}
  \hfill
  %\includegraphics[scale=0.25]{EISCAT3Dlogo1.pdf}
  % New official logo with green text
  \includegraphics[width=0.75\linewidth]{e3d-logo-green-500px}
\end{minipage}

\newpage
\tableofcontents
\newpage

\section{Executive Summary}
\label{exec-summ}

%\todo[inline]{Executive summary to be written at the end of the document writing process? Moved the current exec summary to introduction as that's what it was more like}

Deliverable 4: The synergies with the NeIC NT1 operations. Once decisions on the software to be deployed and run as a simulation of the \ED operations, the possibilities of inclusion to the NT1 operations are studied.The object to be delivered is a document that details the services for \ED operations and those that may or may not be managed by NeIC NT1. This includes the services and service management (e.g. FitSM) required and the division of responsibilities.  Included are estimates of the staffing required to run these services and SLAs needed. 

The study of the synergies between \ED online and NT1 operations may start once the software decisions are made in Deliverable 1 in the sixth month of the project.

The process that follows is continuous and requires input from both \ED and NT1.

Document delivered. See section 3.2.

\section{Purpose}
\label{purpose}

\subsection{Intended Audience}

The intended audience of this document is primarily NeIC and the \ED project management and staff in order to understand 

\section{Introduction}
\label{intro}

\begin{table} \centering
\rowcolors{2}{gray!25}{white}
\begin{tabular}{p{\mycolwidth} p{\mycolwidth} l l l l p{\mycolwidth}}
{\bf \tiny Service}  & {\bf \tiny Location} & \bf {\tiny Real-time} & {\bf \tiny Run by} & \bf {\tiny Owned by} & {\bf \tiny Priority} & {\bf \tiny Comments} \\

%  \bf Service  & \bf Location & \bf Real-time & \bf Run by & \bf Owned by & \bf Priority & \bf Comments
% \bf \tiny Service  & \bf \tiny Location & \bf \tiny Real-time & \bf \tiny Run by & \bf \tiny Owned by & \bf \tiny Priority & \bf \tiny Comments \\
 \tiny Radar controller & \tiny Skibotn or DC & \tiny Yes & \tiny EISCAT & \tiny EISCAT & \tiny 1 High & \tiny Radar controller  system
\\
 \tiny Ring buffer + Beam former + File Writer + Fast Imaging & \tiny Skibotn or all sites & \tiny Yes & \tiny EISCAT & \tiny EISCAT & \tiny 1 High & \tiny RAM and fast disks
\\
 \tiny Prompt computing   & \tiny TBD Skibotn or DC & \tiny Yes & \tiny EISCAT & \tiny EISCAT & \tiny 1 High & \tiny Lag profiling (Level 1 to 2), parameter fitting (Level 2 to 3), visualisation (Level 3 on geographic coordinates)
\\
 \tiny Secondary data products & \tiny Co-located with prompt computing & \tiny No & \tiny Other & \tiny EISCAT & \tiny 2 Medium & \tiny Standard data products that don't need to be produced in realtime e.g. full imaging
\\
 \tiny Identity management & \tiny One master site. Other secondary sites. & \tiny No & \tiny Other & \tiny EISCAT & \tiny 2 Medium & \tiny Directory services for user identity management
\\
 \tiny Site data buffer  & \tiny Co-located with prompt computing & \tiny Yes & \tiny Other & \tiny EISCAT & \tiny 1 High & \tiny Fast disk storage for processing of data, levels 1b, 2, 3. Partition of long-term storage?
\\
 \tiny Cluster management etc & \tiny Sites & \tiny No & \tiny Other & \tiny EISCAT & \tiny 1 High & \tiny Configuration and software deployment on RBBF, Prompt computing, admin login nodes etc. 
\\
 \tiny User portal web GUI & \tiny TBD & \tiny No & \tiny Other & \tiny Other & \tiny 2 Medium & \tiny User data discovery and job submission portal. DIRAC prototype exists.
\\
 \tiny User job management & \tiny TBD & \tiny No & \tiny Other & \tiny Other & \tiny 2 Medium& \tiny Submit analysis jobs. DIRAC considered.
\\
 \tiny Data staging & \tiny Initially Skibotn storage & \tiny No & \tiny Other & \tiny Other & \tiny 2 Medium & \tiny Prototype uses DIRAC storage element and file transfer 
\\
 \tiny User compute & \tiny Initially using Prompt computing & \tiny No & \tiny Other & \tiny EISCAT & \tiny 2 Medium & \tiny Computing resources for user analysis portal.
\\
 \tiny User software deployment & \tiny Complements User analysis portal & \tiny No & \tiny Other & \tiny Other & \tiny 2 Medium & \tiny e.g. the DIRAC input sandbox
\\
 \tiny User software deployment & \tiny Complements User analysis portal & \tiny No & \tiny Other & \tiny Other & \tiny 3 Low & \tiny e.g. Notebooks Juypter or other? 
\\
 \tiny Data archive & \tiny Data Centres & \tiny No & \tiny Other & \tiny EISCAT & \tiny 2 Medium & \tiny Long-term data storage. Does EISCAT want to run a data archive?
\\
 \tiny Internal metadata catalog & \tiny Data Centres & \tiny Yes & \tiny Other & \tiny EISCAT & \tiny 1 High & \tiny Metadata master catalogue e.g. metadata from Level 1b files. Realtime.
\\
 \tiny User metadata catalog & \tiny Data Centres & \tiny No & \tiny Other & \tiny EISCAT & \tiny 2 Medium & \tiny Metadata for users. e.g. from higher level data products.
\\
 \tiny Data product repository & \tiny Close to long-term storage & \tiny No & \tiny Other & \tiny EISCAT & \tiny 2 Medium & \tiny 
\\
 \tiny Data product publishing & \tiny Close to long-term storage & \tiny No & \tiny Other & \tiny EISCAT & \tiny 3 Low & \tiny e.g. FAIR compliant Repository for published datasets. E.g. B2SHARE.
\\
 \tiny Monitoring & \tiny Sites & \tiny Yes & \tiny Other & \tiny EISCAT & \tiny 1 High & \tiny Environment and operational status. Possibly NAV software from Uninett.
\\ 
 \tiny Inventory & \tiny Skibotn & \tiny No & \tiny EISCAT & \tiny EISCAT & \tiny 1 High & \tiny List of all parts (preferably marked with QR codes) and updates to status.
\\
 \tiny Public IP routing & \tiny Sites & \tiny Yes & \tiny NORDUnet  & \tiny NORDUnet & \tiny 3 Low & \tiny Internet access for technical staff and guest instruments.
\\

\end{tabular}
\caption{The \einfra services required for \ED. \label{tab:services}}
\end{table}



\section{\ED Operations}

In these scenarios, the standard assumptions are:
\bitm
\item That the \ED data is replicated to two redundant data archives.
\item Other assumptions...
\eitm
\subsection{Scenario 1}

In this scenario \EC runs all the computing on-line and off-line services noted in Table~\ref{tab:services} except for the second redundant external data archive 
i.e. the copy of the \ED data.

\subsubsection{Risks}

\bitm
\item Large amount of FTEs required to run all services.
\item Recruiting lots of skilled IT staff into EISCAT Scientific Association might fail
\item Estimate of effort? 
\eitm

\subsubsection{Benefits}
\bitm
\item Hardware and staff are controlled through the normal EISCAT processes
\eitm

\section{NT1 Operations}


% end of the document...
\newpage
\bibliography{main}{}
\bibliographystyle{unsrt}


\end{document}